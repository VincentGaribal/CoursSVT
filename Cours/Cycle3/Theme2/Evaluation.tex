\documentclass{Controle}

\begin{document}

\renewcommand{\correction}{non}

\nom{}

\exercice{Archaeopteryx}

\emph{Archaeopteryx}, qui signifie «~aile antique~», est un genre de dinosaure à plumes disparu. Comme \emph{Sinocalliopteryx}, on considère \emph{Archeopteryx} comme un des ancêtres des oiseaux actuels. Comme eux, il possédait des plumes qu'il utilisait pour le vol et il avait un squelette interne. En revanche, il ne possédait pas de bec mais une bouche avec des dents.

\question[]{Complétez le tableau de caractères suivant en mettant une croix lorsque le caractère est présent.}{4}{%
\begin{tabularx}{\linewidth}{|c|*{Y|}}
\hline
Caractère & \emph{Archaeopteryx} & Renard & Canard colvert & Carabe violet \\ \hline
Squelette interne & & & & \\ \hline
Squelette externe & & & & \\ \hline
Plume & & & & \\ \hline
Poils & & & & \\ \hline
\end{tabularx}
}{%
\begin{tabularx}{\linewidth}{|c|*4{Y|}}
\hline
Caractère & \emph{Archaeopteryx} & Renard & Canard colvert & Carabe violet \\ \hline
Squelette interne & & & & \\ \hline
Squelette externe & & & & \\ \hline
Plume & & & & \\ \hline
Poils & & & & \\ \hline
\end{tabularx}}

\question[]{à partir du tableau de caractères de la question 1, complétez la classification en groupes emboîtés ci dessous en mettant les noms des groupes suivants :
  \begin{itemize}
  \item Mammifères (possèdent des poils);
  \item Oiseaux (possèdent des plumes);
  \item Vertébrés (possèdent un squelette interne);
  \item Arthropodes (possèdent un squelette externe).
  \end{itemize}
  Attention, chaque boîte, quelle que soit sa taille, ne peut avoir qu'un seul nom.}{4}{}{}

\noindent
\begin{tikzpicture}
  \draw (0,0) rectangle (20,5);
  \draw (0.5,0.5) rectangle (12,4);
  \draw (12.5,0.5) rectangle (19.5,4);
  \draw (1,1) rectangle (5.75,3);
  \draw (6.25,1) rectangle (11.5,3);
\end{tikzpicture}

  
\end{document}