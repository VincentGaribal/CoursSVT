\documentclass{Controle}

\begin{document}

\nom

\section*{Une blessure en EPS}

Lors d'un entrainement à la course, un élève se blesse au bras en tombant. La plaie saigne quelques minutes. Le professeur n'ayant rien d'autre à disposition, son professeur entoure la zone blessée avec un bandage.

Deux jours plus tard, l'élève ne se sent pas bien. Il demande à aller à l'infirmerie, et l'infirmière lui dit qu'il a de la fièvre et qu'il doit rentrer chez lui.

L'élève voit son médecin le lendemain qui lui prescrit une prise de sang. Le médecin lui annonce une infection très sérieuse, car le résultat de l'analyse sanguine indique la présence de bactéries \emph{Staphylococcus aureus}, les staphylocoques dorés.

\question{Explique l'origine probable de l'infection de l'élève. Tu définiras le terme «~infection~».}{5}{}{\ligne{3}}

\question{Quelle est la barrière naturelle qui a été francjie dans ce cas ? Quel mot emploie-t-on pour dire qu'une barrière naturelle a été franchie ?}{4}{}{\ligne{4}}

\question{D'après tes connaissances, quel adjectif peut qualifier la bactérie \emph{Staphylococcus aureus} reperée par l'analyse de sang ?}{3}{}{\ligne{3}}

\question{Quelles cellules du système immunitaire ont été imméditement mobilisée pour lutter contre les bactéries \emph{Staphylococcus aureus} après leur entrée dans le corps ? Comment s'appelle l'action de ces cellules ? Ont-elles réussi à combattre ces bactéries ?}{5}{}{\ligne{5}}



\end{document}