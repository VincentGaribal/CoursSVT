\documentclass[a4paper,landscape,8pt]{extreport}
\usepackage{luatex85}
\usepackage[margin=5mm]{geometry}
\usepackage{fontspec}
\setmainfont[BoldFont={Source Sans Pro Semi Bold}]{Source Sans Pro Light}
\usepackage{booktabs}
\usepackage{tabularx}
\usepackage[french]{babel}
\usepackage{microtype}
\usepackage{hyperref}
\hypersetup{%
  unicode=true%
}

\title{Progression commune de SVT : cycle 4}
\author{Collège Elsa Triolet}
\date{dernière mise à jour le \today}

\begin{document}
\maketitle

\phantomsection
\addcontentsline{toc}{chapter}{Cinquième}
\phantomsection
\addcontentsline{toc}{section}{Thème 1 : la planète Terre, l'environnement et l'action humaine}
\phantomsection
\addcontentsline{toc}{subsection}{La Terre}

\noindent
\begin{tabularx}{\linewidth}{p{3cm}p{3cm}cp{6cm}*3{X}}
  \toprule
  \textbf{Chapitre} & \textbf{Leçon} & \textbf{Durée} & \textbf{Idées clés} & \textbf{Connaissances} & \textbf{Compétences} & \textbf{Exemple d'activité} \\
  \midrule
  La Terre, une planète particulière & Le système solaire & 1h & Expliquer ce que la Terre a de spécifique et ce qu’elle partage avec différents objets du système solaire & Le système solaire, les planètes telluriques et les planètes gazeuses & & Approche historique \\
  \cmidrule(l){2-7}
                    & La Terre, une planète en mouvement & 1h & Expliquer le rôle majeur du Soleil sur certaines des caractéristiques des planètes telluriques et gazeuses & Le globe terrestre (forme, rotation) & & \\
  \cmidrule(l){2-7}
                    & La Terre, une planète active & 2h & & Le globe terrestre (séismes, éruptions volcaniques) & & \\
  \midrule
\end{tabularx}

\end{document}