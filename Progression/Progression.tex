\documentclass{Progression}

\title{Progression commune de SVT : cycle 4}
\author{Collège Elsa Triolet}
\date{dernière mise à jour le \today}

\begin{document}

\maketitle

\phantomsection
\addcontentsline{toc}{chapter}{Cinquième}
\phantomsection
\addcontentsline{toc}{section}{Thème 1 : La planete Terre, l’environnement et l’action humaine}
\phantomsection
\addcontentsline{toc}{subsection}{La Terre}

\noindent
\begin{tabularx}{\linewidth}{p{3cm}p{3cm}cp{6cm}*3{X}}
  \toprule
  \textbf{Chapitre} & \textbf{Leçon} & \textbf{Durée} & \textbf{Idées clés} & \textbf{Connaissances} & \textbf{Compétences} & \textbf{Exemple d'activité} \\
  \midrule
  La Terre, une planète particulière & Le système solaire & 1h & Expliquer ce que la Terre a de spécifique et ce qu’elle partage avec différents objets du système solaire & Le système solaire, les planètes telluriques et les planètes gazeuses & & Approche historique \\
  \cmidrule(l){2-7}
                    & La Terre, une planète en mouvement & 1h & Expliquer le rôle majeur du Soleil sur certaines des caractéristiques des planètes telluriques et gazeuses & Le globe terrestre (forme, rotation) & & \\
  \cmidrule(l){2-7}
                    & La Terre, une planète active & 2h & & Le globe terrestre (séismes, éruptions volcaniques) & & \\
  \midrule
\end{tabularx}

\clearpage

\phantomsection
\addcontentsline{toc}{section}{Thème 2 : Le vivant et son évolution}
\subsection*{La nutrition chez les animaux}
\phantomsection
\addcontentsline{toc}{subsection}{La nutrition chez les animaux}

\noindent
\begin{tabularx}{\linewidth}{p{3cm}p{3cm}cp{6cm}*3{X}}
  \toprule
  \textbf{Chapitre} & \textbf{Leçon} & \textbf{Durée} & \textbf{Idées clés} & \textbf{Connaissances} & \textbf{Compétences} & \textbf{Exemple d'activité} \\  \midrule
  La nutrition chez les animaux & Quelles sont les caractéristiques de la nutrition animale ? & 1h & Relier des systèmes digestifs à des régimes alimentaires (phytophages ; zoophages) & Nutrition et organisation fonctionnelle à l’échelle de l’organisme, des organes, des tissus et des cellules & Relier les besoins des cellules animales et le rôle des systèmes de transport dans l’organisme & Exemple de régimes alimentaires (réseaux alimentaires) \\
  \cmidrule(l){2-4}
  \cmidrule(l){7-7}
  & Quels sont les différents systèmes d’approvisionnement en dioxygène ? & 2h & Relier le passage du dioxygène des milieux de vie au niveau des appareils respiratoires aux caractéristiques des surfaces d’échanges & & & Dissection branchies, poumons\\
  \cmidrule(l){2-4}
  \cmidrule(l){7-7}
  & Comment le dioxygène et les nutriments sont-ils amenés aux organes ? & 2h & Relier les systèmes de transport (appareil circulatoire endigué ou non ; milieu intérieur) aux lieux d’utilisation et de stockage des nutriments (besoins des cellules ; tissus de stockage) & & & Dissection cœur, expériences historiques Harvey\\
  \cmidrule(l){2-4}
  \cmidrule(l){7-7}
  & Comment les déchets de l’organisme sont-ils éliminés ? & 1h & Relier les systèmes de transport et l’élimination des déchets produits au cours du fonctionnement cellulaire & & & \\
  \cmidrule(l){2-5}
  \cmidrule(l){7-7}
  & Quel est le rôle des micro-organismes dans la digestion ? & 1h & Relier la présence de micro-organismes dans le tube digestif à certaines caractéristiques de la digestion & Nutrition et interactions avec des micro-organismes & & \\
  \bottomrule
\end{tabularx}

\subsection*{La reproduction des êtres vivants}
\phantomsection
\addcontentsline{toc}{subsection}{La reproduction des êtres vivants}

\noindent
\begin{tabularx}{\linewidth}{p{3cm}p{3cm}cp{6cm}*3{X}}
  \toprule
  \textbf{Chapitre} & \textbf{Leçon} & \textbf{Durée} & \textbf{Idées clés} & \textbf{Connaissances} & \textbf{Compétences} & \textbf{Exemple d'activité} \\  \midrule
  La reproduction des êtres vivants & Comment s’effectue la reproduction des animaux ? & 1h & \multirow{2}{=}{Relier certaines modalités de la reproduction sexuée (oviparité/viviparité ; fécondation externe/interne ; reproduction des plantes à fleurs) aux pressions exercées par les milieux} & \multirow{2}{=}{Reproductions sexuée et asexuée, rencontre des gamètes, milieux et modes de reproduction} & \multirow{2}{=}{Relier des éléments de biologie de la reproduction sexuée et asexuée des êtres vivants et l’influence du milieu sur la survie des individus, à la dynamique des populations} & Exemples dans différents milieux (aquatique, terrestre) \\
  \cmidrule(l){2-3}
  \cmidrule(l){7-7}
  & Comment s’effectue la reproduction des plantes à fleurs ? & 1h & & & & Dissection de fleur ; pollinisation \\
  \cmidrule(l){2-4}
  \cmidrule(l){7-7}
  & Comment les plantes à fleurs se reproduisent-elles sans graines ? & 1h & Identifier des modes de reproduction asexuée & & & Exemple naturel (division de rhizome et plantation de tubercule). Évoquer le bouturage.\\
  \bottomrule
\end{tabularx}

\end{document}