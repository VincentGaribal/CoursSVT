\documentclass{Programmation}
\usepackage{bigstrut}

\title{Programmation de SVT}
\author{Collège Elsa Triolet}
\date{dernière mise à jour : \today}

\begin{document}

\maketitle

\section*{Légende}
\subsection*{Thème 1}
\begin{tabularx}{9cm}{p{1.5cm}X}
  \hline
  \Terre & Terre\\ \hline
  \SVTG & Séismes, volcanisme, tectonique globale\\ \hline
  \Eres & Ères géologiques\\ \hline
  \MC & Météo et climat\\ \hline
  \Risque & Risques naturels\\ \hline
  \Ress & Exploitation des ressources naturelles\\ \hline
\end{tabularx}

\subsection*{Thème 2}
\begin{tabularx}{9cm}{p{1.5cm}X}
  \hline
  \NutVeg & Nutrition végétale\\ \hline
  \NutAni & Nutrition animale\\ \hline
  \Repro & Reproductions animale et végétale\\ \hline
  \Genet & Génétique\\ \hline
  \Evo & Évolution\\ \hline
\end{tabularx}

\subsection*{Thème 3}
\begin{tabularx}{9cm}{p{1.5cm}X}
  \hline
  \Act & Activités musculaire, nerveuse, cardiovasculaire et cérébrale\\ \hline
  \Dig & Digestion\\ \hline
  \Micro & Micro-organismes\\ \hline
  \Immuno & Immunologie\\ \hline
  \RepHum & Reproduction humaine\\ \hline
\end{tabularx}

\section*{Programmation 2017-- 2018 résumée}
\subsection*{Cinquième}
\begin{tabular}{*{9}{p{1.5cm}}}
  \hline
  \Terre&\MC    &\NutAni&\Micro&\Evo&\Act&\RepHum & & \\ \hline
  \Terre&\Ress  &\NutAni&\Repro&\Evo&\Dig&\RepHum & & \\ \hline
  \SVTG &\Ress  &\NutAni&\Repro&\Act&\Dig& & & \\ \hline
  \SVTG &\Ress  &\NutAni&\Repro&\Act&\Dig& & & \\ \hline
  \MC   &\NutAni&\NutAni&\Evo  &\Act&\Dig& & & \\ \hline
\end{tabular}

\subsection*{Quatrième}
\begin{tabular}{*{9}{p{1.5cm}}}
  \hline
  \SVTG&\Risque&\Ress & & & & & & \\ \hline
  \SVTG&\Risque& & & & & & & \\ \hline
  \SVTG&\Ress  & & & & & & & \\ \hline
  \MC  &\Ress  & & & & & & & \\ \hline
  \MC  &\Ress  & & & & & & & \\ \hline
\end{tabular}

\subsection*{Troisième}
\begin{tabular}{*{9}{p{1.5cm}}}
  \hline
  & & & & & & & & \\ \hline
  & & & & & & & & \\ \hline
  & & & & & & & & \\ \hline
  & & & & & & & & \\ \hline
  & & & & & & & & \\ \hline
\end{tabular}

\clearpage

\section*{Programmation 2017--2018 détaillée}

\subsection*{Thème 1 : la planète Terre, l'environnement et l'action humaine}

\begin{tabularx}{\linewidth}{p{2cm}*2{X}*3{c}}
  \toprule
  \textbf{Sous thème} & \textbf{Compétence} & \textbf{Notion} & \textbf{5\ieme} & \textbf{4\ieme} & \textbf{3\ieme}\\
  \midrule
  La Terre dans le système solaire & \multirow{2}{=}{Expliquer quelques phénomènes géologiques à partir du contexte géodynamique global} & Le système solaire, les planètes telluriques et les planètes gazeuses & \themeuncinq{1} & & \\
  \cmidrule(l){3-6}
  & & Le globe terrestre : forme, rotation & \themeuncinq{1} & & \\
  \cmidrule(l){3-6}
  & & Dynamique interne et tectonique des plaques & & \themeunquatre{5} & \\
  \cmidrule(l){3-6}
  & & Séismes & & \themeunquatre{3} & \\
  \cmidrule(l){3-6}
  & & Éruptions volcaniques & \themeuncinq{4} & & \\
  \cmidrule(l){3-6}
  & & Les ères géologiques & & & \themeuntrois{1} \\
  \midrule
  \multirow{2}{=}{Phénomènes météorologiques et climatiques} & \multirow{2}{=}{Expliquer quelques phénomènes météorologiques et climatiques} & Météorologie & \themeuncinq{1} & & \\
  \cmidrule(l){3-6}
  & & Dynamique des masses d’air et des masses  d’eau ; vents et courants océaniques & \themeuncinq{3} & & \\
  \cmidrule(l){3-6}
  & & Différence entre météo et climat & \themeuncinq{1} & & \\
  \cmidrule(l){3-6}
  & & Les grandes zones climatiques de la Terre & \themeuncinq{1} & & \\
  \cmidrule(l){3-6}
  & & Les changements climatiques passés (temps géologiques) et actuels (influence des activités humaines sur le climat) & & & \themeuntrois{1} \\
  \midrule
  Risques naturels & \multirow{2}{=}{Relier les connaissances scientifiques sur les risques naturels (ex. séismes, cyclones, inondations) ainsi que ceux liés aux activités humaines (pollution de l’air et des mers, réchauffement climatique…) aux mesures de prévention (quand c’est possible), de protection, d’adaptation, ou d’atténuation}  & Les phénomènes naturels : risques et enjeux pour l’être humain & \themeuncinq{2} & \themeunquatre{2} & \\
  & & & & & \\
  & & & & & \\
  \cmidrule(l){3-6}
  & & Notions d’aléas, de vulnérabilité et de risque en lien avec les phénomènes naturels ; prévisions & & \themeunquatre{2} & \\
  \midrule
  Exploitation des ressources naturelles et action humaine & Caractériser quelques-uns des principaux enjeux de l’exploitation d’une ressource naturelle par l’être humain, en lien avec quelques grandes questions de société.	& L’exploitation de quelques ressources naturelles par l’être humain (eau, sol, pétrole, charbon, bois, ressources minérales, ressources halieutiques, etc.) pour ses besoins en nourriture et ses activités quotidiennes & \themeuncinq{2} & \themeunquatre{2} & \\
  \cmidrule(l){2-6}
  & Comprendre et expliquer les choix en matière de gestion de ressources naturelles à différentes échelles & & \themeuncinq{1} & & \\
  \cmidrule(l){2-6}
  & Expliquer comment une activité humaine peut modifier l’organisation et le fonctionnement des écosystèmes en lien avec quelques questions environnementales globales & & \themeuncinq{1} & & \\
  \cmidrule(l){2-6}
  & Proposer des argumentations sur les impacts générés par le rythme, la nature (bénéfices/nuisances), l’importance et la variabilité des actions de l’être humain sur l’environnement & Quelques exemples d’interactions entre les activités humaines et l’environnement, dont l’interaction être humain - biodiversité (de l’échelle d’un écosystème local et de sa dynamique jusqu’à celle de la planète) & \themeuncinq{1} & & \\
  \midrule
  \multicolumn{2}{l}{\textbf{Total pour le thème par niveau}} & & \textbf{\thecinquiemetotal} & \textbf{\thequatriemetotal} & \textbf{\thetroisiemetotal} \\
  \midrule
  \multicolumn{2}{l}{\textbf{Total pour le thème pour le cycle}} & & \multicolumn{3}{c}{\textbf{\thethemeuntotal}}\\
  \bottomrule
\end{tabularx}

\setcounter{cinquiemetotal}{0}
\setcounter{quatriemetotal}{0}
\setcounter{troisiemetotal}{0}

\clearpage

\subsection*{Thème 2 : Le vivant et son évolution}

\begin{tabularx}{\linewidth}{p{2cm}*2{X}|*3{c|}}
  \toprule
  \textbf{Sous thème} & \textbf{Compétence} & \textbf{Notion} & \textbf{5\ieme} & \textbf{4\ieme} & \textbf{3\ieme}\\
  \midrule
  La nutrition des organismes & Relier les besoins des cellules animales et le rôle des systèmes de transport dans l’organisme & Nutrition et organisation fonctionnelle à l’échelle de l’organisme, des organes, des tissus et des cellules & \themedeuxcinq{4} & & \\
  \cmidrule(l){3-6}
  & & Nutrition et interactions avec des micro-organismes & \themedeuxcinq{1} & & \\
  \cmidrule(l){2-6}
  & \multirow{2}{=}{Relier les besoins des cellules d’une plante chlorophyllienne, les lieux de production ou de prélèvement de matière et de stockage et les systèmes de transport au sein de la plante} & Nutrition et organisation fonctionnelle à l’échelle de l’organisme, des organes, des tissus et des cellules & & \themedeuxquatre{2} & \\
  & & & & & \\
  \cmidrule(l){3-6}
  & & Nutrition et interactions avec des micro-organismes & & \themedeuxquatre{1} & \\
  \midrule
  \multirow{2}{=}{La reproduction et la dynamique des populations} & \multirow{3}{=}{Relier des éléments de biologie de la reproduction sexuée et asexuée des êtres vivants et l’influence du milieu sur la survie des individus, à la dynamique des populations} & Reproductions sexuée et asexuée, rencontre des gamètes & \themedeuxcinq{2} & & \\
  \cmidrule(l){3-6}
  & & Milieux et modes de reproduction & \themedeuxcinq{2} & & \\
  \cmidrule(l){3-6}
  & & Gamètes et patrimoine génétique chez les Vertébrés et les plantes à fleurs & & & \themedeuxtrois{2} \\
  \midrule
  Génétique & \multirow{2}{=}{Expliquer sur quoi reposent la diversité et la stabilité génétique des individus} & ADN, mutations & & \themedeuxquatre{1} & \themedeuxtrois{4} \\
  \cmidrule(l){3-6}
  & & Mitose & \themedeuxquatre{2} & & \\
  \cmidrule(l){3-6}
  & & Brassage, gène, méïose et fécondation & & \themedeuxquatre{2} & \themedeuxtrois{4} \\
  \cmidrule(l){2-6}
  & Expliquer comment les phénotypes sont déterminés par les génotypes et par l’action de l’environnement & Diversité génétique au sein d’une population ; héritabilité, stabilité des groupes & & & \themedeuxtrois{2} \\
%	Relier, comme des processus dynamiques, la diversité génétique et la biodiversité.	Diversité et dynamique du monde vivant à différents niveaux d’organisation;    
%		Diversité des relations interspécifiques.
%Evolution	Relier l’étude des relations de parenté entre les êtres vivants, et l’évolution.	Caractères partagés et classification.
%		Les grands groupes d’êtres vivants, dont Homo sapiens, leur parenté et leur évolution.
%	Mettre en évidence des faits d’évolution des espèces et donner des arguments en faveur de quelques mécanismes de l’évolution.	Apparition et disparition d’espèces au cours du temps (dont les premiers organismes vivants sur Terre).
%		Maintien des formes aptes à se reproduire, hasard, sélection naturelle.
  \multicolumn{2}{l}{\textbf{Total pour le thème par niveau}} & & \textbf{\thecinquiemetotal} & \textbf{\thequatriemetotal} & \textbf{\thetroisiemetotal} \\
  \midrule
  \multicolumn{2}{l}{\textbf{Total pour le thème pour le cycle}} & & \multicolumn{3}{c|}{\textbf{\thethemedeuxtotal}}\\
  \bottomrule
\end{tabularx}

\end{document}